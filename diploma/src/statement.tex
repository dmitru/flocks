
\chapter{Цель работы}
\todo{дополнить, сменить прошедшее время на настоящее}
В данной работе рассматривается задача о движении агентов в определенной формации. Изучается нелинейная модель движения агентов на плоскости, описанная в работе \cite{veerman2005flocks}. Оригинальная модель была расширена, чтобы сделать ее более реалистичной и в уже получившейся модели эмпирически при помощи компьютерной симуляции изучаются зависимости скорости сходимости к формации от параметров модели. 

Показана сильная зависимость скорости сходимости модели от алгебраической связности графа коммуникации. Для случая неориентированных графов коммуникации предложен эвристический рандомизированный алгоритм построения графов с определенным числом ребер с большими значениями алгебраической связности (относительно случайного графа с тем же числом ребер). В экспериментах с моделью показано, что графы коммуникации, порожденные этим алгоритмом, дают в среднем б\emph{о}льшую скорость сходимости чем случайно построенный граф коммуникации.

К целям работы также относится создание интерактивной среды для проведения дальнейших экспериментов с подобными моделями, возможно уже другими исследователями. Все исходные коды как среды, так и самой работы выложены в открытый репозиторий по ссылке:
\url{https://github.com/dmitru/flocks}.

\clearpage