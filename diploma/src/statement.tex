
\chapter{Цель работы}
В данной работе рассматривается задача о движении агентов в определенной формации. Изучается нелинейная модель движения агентов на плоскости, описанная в работе \cite{veerman2005flocks}. Оригинальная модель расширяется, чтобы сделать ее более реалистичной и в уже получившейся модели эмпирически при помощи компьютерной симуляции изучаются зависимости скорости сходимости к формации от различных параметров модели. 

Далее в работе изучается связь между скоростью сходимости и свойствами графа влияний. В частности, изучается эффект от увеличения числа ребер в графе влияний на скорость сходимости. Ставится задача о связи свойств графа и скорости сходимости при фиксированном числе ребер. Эта связь изучается для трех общих групп графов и делается вывод о том, какие графы обеспечивают наилучшую сходимость при данном числе ребер. Этот вывод может служить практической рекомендацией для создателей многоагентных систем.

\clearpage