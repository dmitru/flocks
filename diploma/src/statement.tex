
\chapter{Цель работы}
В данной работе рассматривается задача о движении агентов в определенной формации. Изучается нелинейная модель движения агентов на плоскости, описанная в работе \cite{veerman2005flocks}. Оригинальная модель расширяется, чтобы сделать ее более реалистичной и в уже получившейся модели эмпирически при помощи компьютерной симуляции изучаются зависимости скорости сходимости к формации от различных параметров модели. 

Далее в работе изучается связь между скоростью сходимости и свойствами графа влияний. В частности, изучается эффект от изменения числа ребер в графе влияний и от его алгебраической связности. Для фиксированного числа ребер изучается три общих класса графов и делается вывод о том, какие графы обеспечивают лучшую сходимость к формации.

К целям работы также относится создание интерактивной среды для проведения дальнейших экспериментов с подобными моделями, возможно уже другими исследователями. Все исходные коды как среды, так и самой работы выложены в открытый репозиторий по ссылке:
\url{https://github.com/dmitru/flocks}.

\clearpage