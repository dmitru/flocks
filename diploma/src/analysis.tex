\chapter{Интерпретация результатов и выводы} \label{analysis}

\subsection*{Эксперименты с параметрами модели}

\subsection*{Эксперименты с графами влияния}

Анализируя результаты экспериментов с изменением числа ребер и различными типами графов влияний, можно сделать ряд выводов. 

\begin{enumerate}
\item Увеличение числа ребер графа~--- хороший способ повышения скорости сходимости.  Но это действенно только до некоторого предела (см. график \ref{img:graphs-edges}). Когда ребер становится больше чем приблизительно $2/3$ от максимального количества, дальнейшее увеличение их числа практически не влияет на скорость сходимости.

\item При фиксированном числе ребер скорость сходимости для графов с неоднородными степенями вершин в среднем выше чем для графов, однородных по степеням вершин (см. графики \ref{img:graphs-comparison-1}). Этот вывод может служить рекомендацией при выборе топологии графа влияний для построения реальных систем.

\item Не смотря на то, что при фиксированном числе ребер между скоростью сходимости и алгебраической связностью определенно существует положительная корреляция, максимизации одной только алгебраической связности не хватает для достижения наилучшей сходимости. По всей видимости, имеются и другие параметры помимо алгебраической связности, оказывающие влияние на скорость сходимости. На графиках \ref{img:graphs-comparison-1} показано, как средняя алгебраическая связность однородных по степеням вершин графов больше, чем для неоднородных графов, а для скорости сходимости зависимость обратная. 

На графике \ref{img:graphs-comparison} также видно, что при данном значении алгебраической связности графы неоднородной структуры позволяют достичь б\emph{о}льшей скорости сходимости в сравнении с однородными графами, особенно когда ребер немного. Можно выдвинуть предположение о том, существует некий графовый параметр, принимающий различные значения в этих двух случаях и влияющий и сорость сходимости.
\end{enumerate}

\clearpage
