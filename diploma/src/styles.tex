
%%% Макет страницы %%%
% Выставляем значения полей (ГОСТ 7.0.11-2011, 5.3.7)
\geometry{a4paper,top=2cm,bottom=2cm,left=2.5cm,right=1cm}

\def\wl{\par \vspace{\baselineskip}}

%%% Кодировки и шрифты %%%
\ifxetex
  \setmainlanguage{russian}
  \setotherlanguage{english}
  \defaultfontfeatures{Ligatures=TeX,Mapping=tex-text}
  \setmainfont{Times New Roman}
  \newfontfamily\cyrillicfont{Times New Roman}
  \setsansfont{Arial}
  \newfontfamily\cyrillicfontsf{Arial}
  \setmonofont{Courier New}
  \newfontfamily\cyrillicfonttt{Courier New}
\else
  \IfFileExists{pscyr.sty}{\renewcommand{\rmdefault}{ftm}}{}
\fi

%%% Интервалы %%%
\linespread{1.3}                    % Полуторный интвервал (ГОСТ Р 7.0.11-2011, 5.3.6)

%%% Выравнивание и переносы %%%
\sloppy                             % Избавляемся от переполнений
\clubpenalty=10000                  % Запрещаем разрыв страницы после первой строки абзаца
\widowpenalty=10000                 % Запрещаем разрыв страницы после последней строки абзаца

%%% Библиография %%%
\makeatletter
\bibliographystyle{utf8gost780u}     % Оформляем библиографию по ГОСТ 7.1 (ГОСТ Р 7.0.11-2011, 5.6.7)
\renewcommand{\@biblabel}[1]{#1.}   % Заменяем библиографию с квадратных скобок на точку
\makeatother

%%% Изображения %%%
\graphicspath{{images/}}            % Пути к изображениям

%%% Цвета гиперссылок %%%
\definecolor{linkcolor}{rgb}{0.9,0,0}
\definecolor{citecolor}{rgb}{0,0.6,0}
\definecolor{urlcolor}{rgb}{0,0,1}
\hypersetup{
    colorlinks, linkcolor={linkcolor},
    citecolor={citecolor}, urlcolor={urlcolor}
}

%%% Оглавление %%%
\renewcommand{\cftchapdotsep}{\cftdotsep}

%%% Шаблон %%%
\newcommand{\todo}[1]{\textcolor{red}{#1}}

%%% Списки %%%
% Используем дефис для ненумерованных списков (ГОСТ 2.105-95, 4.1.7)
\renewcommand{\labelitemi}{\normalfont\bfseries{--}} 

%%% Колонтитулы %%%
% Порядковый номер страницы печатают на середине верхнего поля страницы (ГОСТ Р 7.0.11-2011, 5.3.8)
\makeatletter
\let\ps@plain\ps@fancy              % Подчиняем первые страницы каждой главы общим правилам
\makeatother
\pagestyle{fancy}                   % Меняем стиль оформления страниц
\fancyhf{}                          % Очищаем текущие значения
\fancyhead[C]{\thepage}             % Печатаем номер страницы на середине верхнего поля
\renewcommand{\headrulewidth}{0pt}  % Убираем разделительную линию

% Векторы жирным шрифтом
\renewcommand{\vec}[1]{\mathbf{#1}}
\renewcommand{\hat}[1]{\oldhat{\mathbf{#1}}}

% Векторы-столбцы
\makeatletter
\newcommand{\veccol}[2][r]{%
  \gdef\@VORNE{1}
  \left(\hskip-\arraycolsep%
    \begin{array}{#1}\vekSp@lten{#2}\end{array}%
  \hskip-\arraycolsep\right)}

\def\vekSp@lten#1{\xvekSp@lten#1;vekL@stLine;}
\def\vekL@stLine{vekL@stLine}
\def\xvekSp@lten#1;{\def\temp{#1}%
  \ifx\temp\vekL@stLine
  \else
    \ifnum\@VORNE=1\gdef\@VORNE{0}
    \else\@arraycr\fi%
    #1%
    \expandafter\xvekSp@lten
  \fi}
\makeatother

% Окружения для теорем. определений и т.д.
\newtheorem{theorem}{Теорема}[section]
\newtheorem{lemma}[theorem]{Лемма}
\newtheorem{proposition}[theorem]{Предложение}
\newtheorem{corollary}[theorem]{Следствие}

\newenvironment{proof}[1][Доказательство]{\begin{trivlist}
\item[\hskip \labelsep {\bfseries #1}]}{\end{trivlist}}

\newcounter{definitionCounter}
\stepcounter{definitionCounter}
\newenvironment{definition}[1][Определение]
{\begin{trivlist}
\item[\hskip \labelsep {\bfseries #1 \arabic{definitionCounter}}]\stepcounter{definitionCounter}}
{\end{trivlist}}

\newenvironment{example}[1][Пример]
{\begin{trivlist}
\item[\hskip \labelsep {\bfseries #1}]}
{\end{trivlist}}

\newenvironment{remark}[1][Замечание]{\begin{trivlist}
\item[\hskip \labelsep {\bfseries #1}]}{\end{trivlist}}