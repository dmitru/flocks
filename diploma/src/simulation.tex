\chapter{Описание системы моделирования} \label{simulation}

В ходе выполения работы была создана компьютерная система для проведения численных экспериментов с рассмотренными моделями. 
Система работает в двух основных режимах:

\section{Интерактивный режим}
Основная задача этого режима - позволить исследователю интерактивно взаимодействовать с моделью, чтобы в процессе этого выработать гипотезы, которые в дальнейшем можно было бы подтвердить более детальным анализом.

В интерактивном режиме пользователь задает начальные параметры модели и в режиме реального времени может наблюдать  ее движение. В этом режиме уравнение движения решается пошагово на маленьких отрезках времени в перерывах между отрисовками состояния модели на экране. 
Благодаря этому есть возможность менять любой параметр в уравнениях системы, например связав изменение параметров с нажатием на определенную клавишу. 

Также можно попросить систему одновременно с отрисовкой модели строить графики произвольных функций от состояния модели (например близость текущего состояния системы к желаемой формации или проекции скорости отдельных агентов), а также любую текстовую информацию (номер шага, количество столкновений, процент активных ребер графа коммуникации).

Реализация этого режима значительно облегчила процесс отладки модели. Например, \todo{TODO}

\section{Режим сбора данных}
\todo{TODO}

\section{Используемые технологии}


\todo{TODO}

\clearpage
