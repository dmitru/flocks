\chapter{Описание системы моделирования} \label{simulation}

В ходе выполения работы была создана компьютерная система для проведения численных экспериментов с рассмотренными моделями. Особенность системы состоит в почти полном разделении трех процессов, общих для компьютерных экспериментов: первичного знакомства исследователя с моделью (развитие интуиции), сбора значительных объемов экспериментальных данных и, наконец, последующей обработки результатов.
Такое разделение имеет много технических и других преимуществ, среди которых большая модульность получившейся системы и более быстрое прохождение цикла от гипотезы до ее подтверждения/отвержения.

Система работает в двух основных режимах: интерактивный и режим сбора данных. Ниже описан каждый из них, после чего рассказано про использовавшиеся технологии.

\section{Интерактивный режим}
Основная задача этого режима - позволить исследователю в режиме реального времени напрямую взаимодействовать с моделью, чтобы в процессе этого выработать гипотезы, которые в дальнейшем можно было бы подтвердить более детальным анализом. Упор здесь сделан не на эффективность вычислений, а на двухстороннюю коммуникацию между исследователем и моделью.

В этом режиме уравнение движения решается пошагово на маленьких отрезках времени, в перерывах между этим текущее состояние модели отрисовывается на экране. Также в этом режиме имеется возможность связать изменение какого-либо параметра с определенной клавишей на клавиатуре и получить возможность менять параметры во время симуляции.

Также есть возможность одновременно с отрисовкой модели строить графики произвольных функций от состояния модели (например, близость текущего состояния системы к желаемой формации или проекции скорости отдельных агентов), а также выводить любую текстовую информацию (номер шага, количество столкновений, процент активных ребер графа коммуникации).

Интерактивный режим позволил лучше понять происходящие в модели процессы. Также он существенно ускорил процесс выдвижения и провеки гипотез. Поэтому реализация похожих возможностей рекомендуется и для других компьютерных систем моделирования.

\section{Режим сбора данных}


\section{Используемые технологии}


\todo{TODO}

\clearpage
