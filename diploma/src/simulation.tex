\chapter{Описание системы моделирования} \label{simulation}

В ходе выполения работы была создана компьютерная система для проведения численных экспериментов с рассмотренными моделями. 
Система работает в двух основных режимах:

\section{Интерактивный режим}
Основная задача этого режима - позволить исследователю интерактивно взаимодействовать с моделью, чтобы в процессе этого выработать гипотезы, которые в дальнейшем можно было бы подтвердить более детальным анализом.

В интерактивном режиме пользователь задает начальные параметры модели и в режиме реального времени может наблюдать  ее движение. В этом режиме уравнение движения решается пошагово на маленьких отрезках времени в перерывах между отрисовками состояния модели на экране. 
Благодаря этому есть возможность менять любой параметр в уравнениях системы, например связав изменение параметров с нажатием на определенную клавишу. 

Также можно попросить систему одновременно с отрисовкой модели строить графики произвольных функций от состояния модели (например близость текущего состояния системы к желаемой формации или проекции скорости отдельных агентов), а также любую текстовую информацию (номер шага, количество столкновений, процент активных ребер графа коммуникации).

Реализация этого режима значительно облегчила процесс отладки модели. Например, 

\section{Режим сбора данных}
Этот режим предназначен для запуска большого числа экспериментов и сохранения данных на диск. Основная идея в том, чтобы
разделить два процесса: прогон экспериментов и сбор данных и их последующая обработка. Это позволит не тратить время при исследовании данных на симуляцию и одновременно более эффективно запускать эксперименты, в том числе и параллельно на вычислительном кластере.

\section{Используемые технологии}
Система реализована на языке программирования Python с использованием математических пакетов \texttt{numpy} для линейной алгебры, \texttt{scipy} для численного решения дифференциальных уравнений и \texttt{networkx} для работы с графами. 

Интерактивный режим был разработан при помощи пакета для визуализации данных \texttt{matplotlib}, которая ко всему прочему позволяет также анимировать графики и реагировать на нажатия клавиш. 

Интереснее режим сбора данных. Как уже говорилось, имеется возможность параллельного запуска большого числа экспериментов с сохранением результатов для последующего анализа. Параллельный запуск выполняется с помощью среды \texttt{ipython}

\todo{TODO}

\clearpage
