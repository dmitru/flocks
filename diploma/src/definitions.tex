\chapter{Описание модели и основные определения} \label{definitions}

\section{Некоторые сведения из алгебраической теории графов}
\todo{TODO}
\section{Базовая линейная модель движения в формации}
Модель описывает систему из $N$ агентов, движущихся в $d$-мерном пространстве ($d\in\{2,3\}$). Перед агентами стоит следующая задача: начав движение из заданной позиции с заданными скоростями, выстроиться в заранее заданную формацию и продолжить движение в ней. 

Каждый агент обладает информацией про его собственные координаты и скорости, а также про координаты и скорости некоторых других агентов, которые входят в его \emph{множество соседей}. Эти множества задаются \emph{графом коммуникации} $G$, в котором проведено ребро $i\rightarrow j$, если агент $i$ получает информацию от агента $j$.

Теперь опишем модель формально.
Состоянием $i$-го агента является  вектор $x_i$
в пространстве $\mathbb{R}^{2d}$:
$$x_i=x^p_i\otimes\veccol{1;0}+x^v_i\otimes
\veccol{0;1}.$$
Здесь $x^p_i$ и $x^v_i$ это пространственное положение и скорость агента, а $\otimes$\ ---\ произведение Кронекера.~\footnote{Таким образом, вектор $x_i$ имеет такой вид (для $d=2$): $\left(x_i, \dot{x}_i, y_i, \dot{y}_i\right)$. Обозначения позаимствованы из работы \cite{veerman2005flocks}.}

Состояние всей системы описывается вектором $x=\left(x_1,x_2,\ldots,x_N\right)^T$. Желаемая формация задается вектором 
$h={\left(h_1,h_2\ldots h_N\right)^T=h_p\otimes\left(1,0\right)}^T\in\mathbb{R}^{2dN}$.

\begin{definition}
\\\\
Говорят, что система движется \emph{в формации}, когда $\exists$ функции $q(t)$, $w(t)$, такие что:
$
x^p(t)-h^p\equiv q(t)\vec{1},\ x^v(t)\equiv w(t)\vec{1}.
$
\\
Говорят, что система \emph{сходится к формации}, когда
$
x^p(t)-h^p-q(t)\vec{1}\rightarrow 0,\ x^v(t)-w(t)\vec{1}\rightarrow 0.
$
\end{definition}

Смысл определения в том, что при движении в формации все агенты имеют одну и ту же скорость, а их позиции совпадают с требуемой формацией с точностью до некоторого смещения.



\section{Нелинейная модель движения в ориентированной формации}
\todo{TODO}

\clearpage
