\chapter{Описание модели и основные определения} \label{definitions}

\section{Некоторые сведения из алгебраической теории графов}
\todo{TODO}
\section{Базовая линейная модель движения в формации}
Модель (будем далее называть ее "модель 1") описывает систему из $N$ агентов, движущихся в $d$-мерном пространстве (в рассматриваемом случае $d=2$). Перед агентами стоит следующая задача: начав движение из заданной позиции с заданными скоростями, выстроиться в заранее заданную формацию и продолжить движение в ней. 

Каждому агенту известна разница своих координат и скоростей с координатами и скоростями некоторых других агентов, которые входят в его \emph{множество соседей}. Эти множества задаются \emph{графом коммуникации} $G$, в котором проведено ребро $i\rightarrow j$, если агент $i$ получает информацию от агента $j$. Каждый агент знает также свою позицию в желаемой конфигурации и позиции его соседей.

Теперь опишем модель формально.
Состоянием $i$-го агента является  вектор $x_i$
в пространстве $\mathbb{R}^{2d}$:
$$x_i=x^p_i\otimes\veccol{1;0}+x^v_i\otimes
\veccol{0;1}.$$
Здесь $x^p_i$ и $x^v_i$ это пространственное положение и скорость агента, а $\otimes$\ ---\ произведение Кронекера.~\footnote{Таким образом, вектор $x_i$ имеет вид (для $d=2$): $\left(x_i, \dot{x}_i, y_i, \dot{y}_i\right)$. Обозначения позаимствованы из работы \cite{veerman2005flocks}.}

Состояние всей системы описывается вектором $x=\left(x_1,x_2,\ldots,x_N\right)^T$. Желаемая формация задается вектором 
$h={\left(h_1,h_2\ldots h_N\right)^T=h_p\otimes\left(1,0\right)}^T\in\mathbb{R}^{2dN}$.

\begin{definition}
\\\\
Говорят, что система движется \emph{в формации}, когда $\exists$ функции $q(t)$, $w(t)$, такие что:
$
x^p(t)-h^p\equiv q(t)\vec{1},\ x^v(t)\equiv w(t)\vec{1}.
$
\\
Говорят, что система \emph{сходится к формации}, когда
$
x^p(t)-h^p-q(t)\vec{1}\rightarrow 0,\ x^v(t)-w(t)\vec{1}\rightarrow 0.
$
\end{definition}

Смысл определения в том, что при движении в формации все агенты имеют одну и ту же скорость, а их позиции совпадают с требуемой формацией с точностью до некоторого смещения.

Уравнение движения агентов имеет в данной модели следующий вид: $\dot{x_i}=A_1x_i+B_1u_i$, где
$$
A_1=\left( \begin{array}{cccc}
0 & 1 & 0 & 0 \\
0 & a_{22} & 0 & a_{24} \\
0 & 0 & 0 & 1 \\
0 & a_{42} & 0 & a_{44} \end{array} \right),\ \ \  
B_1=\[ \left( \begin{array}{cc}
0 & 0 \\
1 & 0 \\
0 & 0 \\
0 & 1 \end{array} \right).
$$

Здесь $u_i$~--- это управляющий сигнал, который еще предстоит ввести. 

Вид матриц $A_1$, $B_1$ объяснен в работе \cite{veerman2005flocks}. Подматрица 
$A_0=\left( \begin{array}{cc}
a_{22} & a_{24} \\
a_{42} & a_{44} \end{array} \right)$ 
матрицы $A_1$ определяет динамику движения формации как целого. Физический смысл чисел $a_{22},a_{24},a_{42},a_{44}$ такой: $a_{24},\ a_{42}$ отвечают за кривизну траектории формации. При $a_{24}=-a_{42}=k$ агенты будут двигаться в формации по окружности определенного радиуса; при $|a_{24}|\neq|a_{42}|$ ~--- по эллипсу, при $a_{24}=a_{42}$~--- по прямой. Числа $a_{22}$ и $a_{44}$ отвечают за ускорение формации вдоль линии текущего курса, при $a_{22}>0,\ a_{44}>0$ ускорение положительное, при $a_{22},\ a_{44}~$ --- отрицательное.

\begin{figure}[h]
  \begin{minipage}[h]{0.32\linewidth}
    \center{\includegraphics[width=\linewidth]{linear_straight.png} \\ а)}
  \end{minipage}
  \hfill
  \begin{minipage}[h]{0.32\linewidth}
    \center{\includegraphics[width=\linewidth]{linear_circle.png} \\ б)}
  \end{minipage}
   \hfill
  \begin{minipage}[h]{0.32\linewidth}
    \center{\includegraphics[width=\linewidth]{linear_variable_k.png} \\ в)}
  \end{minipage}
  \caption{Движение, задаваемое моделью 1. На рисунке a) параметр кривизны движения $k=0$, на рисунке б) $k=-1$, на рисунке в) параметр $k$ меняет значение во время движения. Во всех случаях граф коммуникации полный, $f_1=f_2=-2.5$.}
\label{fig:linear-motion-1}
\end{figure}

Управляющий сигнал получается преобразованием $u=Fz$, где $F$~--- матрица обратной связи, а $z$~---функция, описывающая "выходной сигнал" из системы. Этот выходной сигнал рассчитывается так же как и в базовой модели консенсуса:
$$
z_i=(x_i-h_i)+\frac{1}{|\mathbb{J}_i|}\sum_{j\in\mathbb{J}_i}\left((x_i-h_i)-(x_j-h_j)\right)\ \ \ \ i=1,\ldots,N.
$$

Объединив уравнения для всех агентов в одно, получим общее уравнение движения для всей системы:
\begin{equation}
\dot{x}=Ax+BFL(x-h),
\label{eq:linear-motion}
\end{equation}
где $L\equiv L_G\otimes I_{2N},\ L_G$~---матрица лапласа графа коммуникации $G$, а матрицы $A,\ B$ и $F$ определены через "размножение" матриц $A_1,\ B_1$ и $F_1$ для одного агента: $A=I_N\otimes A_1,\ B=I_N\otimes B_1, F=I_N\otimes F_1$, т.к. законы движения всех агентов одинаковы.

В работах \cite{veerman2005flocks}, \cite{lafferriere2005decentralized} авторы ограничиваются матрицей обратной связи $F_1$ вида 
$$F_1=\left( \begin{array}{cccc}
f_1 & f_2 & 0 & 0 \\
0 & 0 & f_1 & f_2 \end{array} \right),$$ такая же матрица используется и в данной работе.

Для рассмотренной модели имеется ряд доказанных теоретических результатов, основные из которых приведены ниже. 

Во-первых, так же как и в базовой модели консенсуса, существует необходимое условие сходимости модели:
\begin{proposition}
Необходимым условием сходимости модели к формации является наличие в графе коммуникации $G$ остовного исходящего дерева.
\end{proposition}

Но для рассматриваемой модели это условие не является достаточным. Достаточные условия сходимости даются следующей теоремой \cite{veerman2005flocks,lafferriere2005decentralized}:

\begin{theorem}
Модель 1 сходится к формации, когда коэффициенты обратной связи $f_1,\ f_2$ и параметр $k$ удовлетворяют неравенствам 
\begin{equation}
\frac{{f_2}^2}{f_1}< -\frac{\beta^2}{\alpha(\alpha^2+\beta^2)}\ \ \text{и}
\end{equation}
\begin{equation}
|k|<\frac{-f_1|\beta|}{f_2\alpha}-\frac{f_2(\alpha^2+\beta^2)}{|\beta|},
\end{equation}
где $\alpha+i\beta=\lambda$~--- собственные числа матрицы Лапласа графа коммуникации $G$.
\end{theorem}

\section{Нелинейная модель движения в ориентированной формации}
Как можно видеть на рисунке \ref{fig:linear-motion}, формация не меняет своей ориентации при поворотах, что не очень реалистично. Изменив уравнение движения \ref{eq:linear-motion}, получим уже нелинейную модель 2:
\begin{equation}
\dot{x}=Ax+BFL(x-T_x h),
\label{eq:orientable-motion}
\end{equation}
где оператор $T_x$ определен как
\begin{equation}
T_x:h\rightarrow \sum^N_{i=1}E_i\otimes R_{x^v_i}\otimes
\left( 
\begin{array}{cc}
1 & 0 \\
0 & 0 \end{array} \right) h_i.
\end{equation}
В этом определении $E_i$ это матрица, состоящая из нулей и единственной единицей на $i$-й диагональной позиции, а $R_v:\mathbb{R}^2\rightarrow\mathbb{R}^2$ поворачивает базисный вектор $e_1$ в направлении вектора $v$.

Модель значительно усложнилась: перестала быть линейной и от агентов стало требоваться знать не только скорости относительно соседей, но и их собственные асболютные скорости.

На рисунке \ref{img:orientable-motion} показаны траектории агентов в модели 2:

\begin{figure}[h]
  \begin{minipage}[h]{0.45\linewidth}
    \center{\includegraphics[width=\linewidth]{orientable_straight.png} \\ а)}
  \end{minipage}
  \hfill
  \begin{minipage}[h]{0.45\linewidth}
    \center{\includegraphics[width=\linewidth]{orientable_circle.png} \\ б)}
  \end{minipage}
  \caption{Движение, задаваемое моделью 2 --- формация ориентируется по направлению движения. На рисунке б) видно искажение формации во время совершения разворота.}
\label{fig:orientable-motion-1}
\end{figure}

На последнем рисунке видно, что во время разворота агенты выстраиваются не в точности в заданную формацию (на рисунке отмечена квадратами), а в ее немного искаженную версию. Причем при уменьшении радиуса разворота искажение это все больше.

\section{Дальнейшее расширение модели}
\todo{TODO}

\clearpage
