\chapter*{Введение}							% Заголовок
\addcontentsline{toc}{chapter}{Введение}	% Добавляем его в оглавление
Обзор, введение в тему, обозначение места данной работы в мировых исследованиях и т.п.

\textbf{Целью} данной работы является \ldots

Для~достижения поставленной цели необходимо было решить следующие задачи:
\begin{enumerate}
  \item Исследовать, разработать, вычислить и т.д. и т.п.
  \item Исследовать, разработать, вычислить и т.д. и т.п.
  \item Исследовать, разработать, вычислить и т.д. и т.п.
  \item Исследовать, разработать, вычислить и т.д. и т.п.
\end{enumerate}

\textbf{Основные положения, выносимые на~защиту:}
\begin{enumerate}
  \item Первое положение
  \item Второе положение
  \item Третье положение
  \item Четвертое положение
\end{enumerate}

\textbf{Научная новизна:}
\begin{enumerate}
  \item Впервые \ldots
  \item Впервые \ldots
  \item Было выполнено оригинальное исследование \ldots
\end{enumerate}

\textbf{Научная и практическая значимость} \ldots

\textbf{Степень достоверности} полученных результатов обеспечивается \ldots Результаты находятся в соответствии с результатами, полученными другими авторами.

\textbf{Апробация работы.}
Основные результаты работы докладывались~на:
перечисление основных конференций, симпозиумов и т.п.

\textbf{Личный вклад.} Автор принимал активное участие \ldots

\textbf{Публикации.} Основные результаты по теме диссертации изложены в ХХ печатных изданиях~\cite{Sychev,Sokolov,Gaidaenko,Lermontov,Management},
Х из которых изданы в журналах, рекомендованных ВАК~\cite{Sychev,Sokolov,Gaidaenko}, 
ХХ --- в тезисах докладов~\cite{Lermontov,Management}.

\textbf{Объем и структура работы.} Диссертация состоит из~введения, четырех глав, заключения и~двух приложений. Полный объем диссертации составляет ХХХ~страница с~ХХ~рисунками и~ХХ~таблицами. Список литературы содержит ХХХ~наименований.

\clearpage