\section*{Введение}					    % Заголовок
\addcontentsline{toc}{chapter}{Введение}	% Добавляем его в оглавление

Во введении вводится понятие многоагентной системы, описываются преимущества по сравнению с монолитными системами, приводятся примеры удачных приложений и обозначается цель данной работы.

В широком смысле под многоагентной системой понимают
совокупность автономных, т.е. обладающих некоторым 
поведением объектов --- агентов, --- которые кооперируют для решения общей задачи. В процессе решения
задачи агенты могут обмениваться друг с другом
информацией и корректировать свое поведение. 
Ключевыми свойствами многоагентных систем являются:
\begin{itemize}
\item \textbf{Отсутствие центрального управления.} Поведение
агентов определяется ими самими на основе определенных
правил, которые в простейших случаях едины и неизменны
для всех агентов, но в принципе могут и меняться во
времени в зависимости от полученной агентом информации.
\item \textbf{Отсутствие у агентов глобальной информации о всей
системе.} Каждый агент действует, полагаясь на некоторую
"локальную" информацию, полученную из его
непосредственного окружения и от взаимодействия с  соседними агентами.
\end{itemize}

Из этого общего описания вытекают потенциальные преимущества многоагентных систем:
\begin{itemize}
\item Сложность системы и,
 соответственно, сложность решаемых ей задач, определяются 
 не сложностью устройства 
отдельно взятого агента, а их взаимодействием большого числа агентов. 
Поэтому техническое устройство каждого отдельного агента  
может быть достаточно простым, что приводит к \textbf{более низкой стоимости} всей системы и к \textbf{высокой надежности}.
\item \textbf{Высокая отказоустойчивость.} Даже если отдельные агенты вышли из строя, остальные все еще могут выполнить поставленную задачу.
\end{itemize}

В последние годы (начиная с 2004-2005 года) наблюдается
повышенный интерес к теме многоагентных систем и
децентрализованного управления. При этом исследования
не ограничиваются разработкой теории и компьютерными
моделями - в ряде случаев построены успешно
действующие прототипы подобных систем. 

Ниже приведен неполный список задач, в которых успешно  применяется многоагентный подход и ссылки на соответствующие работы:
\begin{itemize}
\item Координация движения различных аппаратов (спутники, беспилотные летатальные аппараты, подводные и наземные средства): \cite{lafferriere2005decentralized}, \cite{veerman2005flocks}, \cite{vasarhelyi2014outdoor}, \cite{williams2005stable}.
\item Сортировка, кластеризация объектов: \cite{deneubourg1991dynamics}, \cite{ding2014sorting}, \cite{kabla2012collective}, \cite{santos2014segregation}.
\item Строительство пространственных структур: \cite{pennisi2014cooperative}, \cite{petersen2014collective}, \cite{augugliaro2013building},\cite{lindsey2011construction}.
\end{itemize}


\clearpage