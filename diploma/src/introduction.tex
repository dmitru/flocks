\section*{Введение}					    % Заголовок
\addcontentsline{toc}{chapter}{Введение}	% Добавляем его в оглавление

В этой части вводится понятие многоагентных систем, описываются их преимущества по сравнению с монолитными системами, приводятся примеры удачных приложений многоагентного подхода и обозначается цель данной работы.

\subsection*{Многоагентные системы}

В широком смысле под многоагентной системой понимают
совокупность автономных, т.е. обладающих некоторым 
поведением объектов --- агентов, --- которые кооперируют для решения общей задачи. В процессе решения
задачи агенты могут обмениваться друг с другом
информацией и на ее основе корректировать свое поведение. 
Ключевыми свойствами многоагентных систем являются:
\begin{itemize}
\item \textbf{Отсутствие центрального управления.} Поведение
агентов определяется ими самими на основе определенных
правил, которые в простейших случаях едины и неизменны
для всех агентов, но в принципе могут и меняться во
времени в зависимости от полученной агентом информации.
\item \textbf{Отсутствие у агентов глобальной информации о всей
системе.} Каждый агент действует, полагаясь на некоторую
"локальную" информацию, полученную из его
непосредственного окружения и от взаимодействия с  соседними агентами.
\end{itemize}

Из этого общего описания вытекают потенциальные преимущества многоагентных систем:
\begin{itemize}
\item Сложность системы и,
 соответственно, сложность решаемых ей задач, определяются 
 не сложностью устройства 
отдельно взятого агента, а взаимодействием большого числа агентов. 
Поэтому техническое устройство каждого отдельного агента  
может быть достаточно простым, что приводит к \textbf{более низкой стоимости} всей системы и к \textbf{повышеной  надежности}.
\item \textbf{Высокая отказоустойчивость.} Даже если отдельные агенты вышли из строя, остальные все еще могут выполнить поставленную задачу.
\end{itemize}

Стоит обратить внимание на сходство многоагентных систем с системами, имеющими место в живой природе. И действительно, некоторые работы строят прямую аналогию с  такими живыми системами как колонии муравьев или стаи птиц и пытаются воспроизвести некоторые модели поведения этих существ.

\subsection*{Некоторые приложения}

В последние годы (начиная с 2004-2005 года) наблюдается
повышенный интерес к теме многоагентных систем и
децентрализованного управления. При этом исследования
не ограничиваются разработкой теории и компьютерными
моделями - в ряде случаев построены успешно
действующие прототипы подобных систем. 

Ниже приведен неполный список задач, в которых успешно  применяется многоагентный подход и ссылки на соответствующие работы:
\begin{itemize}
\item Координация движения различных аппаратов (спутники, беспилотные летатальные аппараты, подводные и наземные средства): \cite{lafferriere2005decentralized}, \cite{veerman2005flocks}, \cite{vasarhelyi2014outdoor}, \cite{williams2005stable}.
\item Сортировка, кластеризация объектов: \cite{deneubourg1991dynamics}, \cite{ding2014sorting}, \cite{kabla2012collective}, \cite{santos2014segregation}.
\item Строительство пространственных структур: \cite{pennisi2014cooperative}, \cite{petersen2014collective}, \cite{augugliaro2013building},\cite{lindsey2011construction}.
\end{itemize}

\subsection*{Цель работы}

В данной работе рассматривается задача о движении агентов в определенной формации. Изучается нелинейная модель движения агентов на плоскости, описанная в работе \cite{veerman2005flocks}. Оригинальная модель была расширена, чтобы сделать ее более реалистичной и в уже получившейся модели эмпирически при помощи компьютерной симуляции изучаются зависимости скорости сходимости к формации от параметров модели. 

Показана сильная зависимость скорости сходимости модели от алгебраической связности графа коммуникации. Для случая неориентированных графов коммуникации предложен эвристический рандомизированный алгоритм построения графов с определенным числом ребер с большими значениями алгебраической связности (относительно случайного графа с тем же числом ребер). В экспериментах с моделью показано, что графы коммуникации, порожденные этим алгоритмом, дают в среднем б\emph{о}льшую скорость сходимости чем случайно построенный граф коммуникации.

К целям работы также относится создание интерактивной среды для проведения дальнейших экспериментов с подобными моделями, возможно уже другими исследователями. Все исходные коды как среды, так и самой работы выложены в открытый репозиторий по ссылке:
\url{https://github.com/dmitru/flocks}.

\clearpage